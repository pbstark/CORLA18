\subsubsection{Combining stratum-level risk limits}\label{sec:stratumRisk}
We audit to test the two hypotheses that $\omega_{w\ell,s} \ge \lambda_s V_{w\ell}$, 
independently for the two strata.
If we reject \emph{both} hypotheses, we conclude that the contest outcome is correct;
otherwise, we manually re-tabulate the contest.

The two audits need to be conducted to smaller risk limits individually than the target overall
risk limit for the contest as a whole.
Recall that the samples are drawn independently from the two strata.
We pick $\alpha_1, \alpha_2 \in (0,\alpha)$.
(More discussion of the choice appears below.)
Also pick $\lambda_1$.
Then if $\omega_{w\ell,1} < \lambda_1 V_{w\ell}$ and 
$\omega_{w\ell,2} < \lambda_2 V_{w\ell}$,
the outcome is correct.
We audit stratum $s$ to test the hypothesis $\omega_s \ge \lambda_s V_{w\ell}$ 
with risk limit $\alpha_s$,
as if it were its own election.
We want to know the relationship between those two stratum-level ``risks'' and the 
overall risk that the audit will not correct the outcome if the outcome is wrong.
That depends in part on what we do if the audit in a given stratum leads to a full manual
tally of that stratum.

Here are some scenarios.
The outcome is certainly correct if both net overstatements are less than their 
respective thresholds. 
For the outcome to be wrong, one or both strata needs to have net overstatement
$\omega_s$
greater than its corresponding threshold $\lambda_s V_{w\ell}$.
If $\omega_1 + \omega_2 \ge V_{w\ell}$, then $\omega_1\ge \lambda_1V_{w\ell}$
or $\omega_2\ge \lambda_2V_{w\ell}$, or both.
If it's only stratum $s$, then the chance that stratum $s$ will be fully hand
counted is at least $1-\alpha_s \ge 1- \alpha$.

If both $\omega_1\ge \lambda_1V_{w\ell}$
and $\omega_2\ge \lambda_2V_{w\ell}$, then the chance both are fully tabulated is
$1-(1-\alpha_1)(1-\alpha_2)$, since the audit samples in the two strata are independent.

What should we do if the audit leads to a full tally in one stratum?
The simplest solution is to require a full hand count of the other stratum, to set the record straight.
If this is the rule, then we can take $\alpha_1 = \alpha_2 = \alpha$, and the procedure will have
risk limit $\alpha$.

Alternatively, we might adjust the contest margin for the ``known'' vote tally in the
fully counted stratum (call the stratum $t$), and continue to audit in the other, but against the entire adjusted
margin $V_{w\ell} - A_{w\ell,t} \equiv \lambda_s' V_{w\ell}$, rather than 
against the share $\lambda_s V_{w\ell}$.
Then to reject the null hypothesis in that stratum means that the overall outcome is still correct.

The wrinkle is that treating the votes
in the fully counted stratum as known changes the hypothesis being tested in a way that is itself random:
whether the original null or a new null is tested depends on what the sample in the other stratum
shows.
(However, if the hypothesis changes, there's only one value possible for the new $\lambda_s$---which
depends on the reported margin and the count in the other stratum---but it's unknown 
until the other stratum count is known.)

The solution is through conditioning. The samples in the two strata are independent. 
Think of the overall procedure as concluding that the outcome (without a full
hand count in both strata) if:

\begin{itemize}
   \item the original hypotheses are rejected in both strata
   \item conditionally on escalating to a full count in one stratum, the threshold $\lambda_s$
            is adjusted in the other stratum, and the hypothesis that the overstatement error
            in that stratum is greater than the new limit, $\lambda_s' V_{w\ell}$ is rejected. 
\end{itemize}

The value of $\lambda_s'$ is a fixed but unknown before the audit starts.
Consider the conditional probability that a sequential test would reject the hypothesis that the margin is less than $\lambda_s' V_{w\ell}$, given that the other stratum goes to a full count. 
Because the tests in the two strata are independent, that's the same as the unconditional probability. 
If we are using a sequential test in the remaining stratum $s$, the chance of ever rejecting the 
hypothesis is at most $\alpha_s$ if the (new) null is true. 
The conditioning just ``delays'' looking at the value of the test statistic for the new null hypothesis; waiting does not increase the overall chance of incorrectly rejecting the null, 
because the test is legitimately sequential.